\documentclass{assignment}
\UsingEnglish
\ProjectInfos*{Intro to Communication System}{EE140}{Fall, 2020}{Assignment 1}{Due time : 10:15, Sept 18, 2020 (Friday)}{陈稼霖}{45875852}
\begin{document}
\begin{prob}[15 pts]
    Classify each of the following signals as an energy signal or as a power signal by calculating E (energy) or P (power). Note: the parameters involved are positive constants.
    \begin{itemize}
        \item[a)] $x(t)=e^{-\alpha\abs{t}}\cos\pi t$,
        \item[b)] $x(t)=\Pi(t-3)\cos 3\pi t$, $\left(\Pi(t)=\left\{\begin{array}{ll}1,&\abs{t}\leq 0.5\\0,&\text{otherwise}\end{array}\right.\right)$
        \item[c)] $x(t)=\abs{t}$,
        \item[d)] $x(t)=\sum_{n=-\infty}^{\infty}\Lambda(\frac{t}{2}-n)$, $\left(\Lambda(t)=\left\{\begin{array}{ll}1-\abs{t},&\abs{t}\\0,&\text{otherwise}\end{array}\right.\right)$
        \item[e)] $x(t)=e^{j2\pi 3t}u(t)$, $\left(u(t)=\left\{\begin{array}{ll}1,&t\geq 0\\0,&\text{otherwise}\end{array}\right.\right)$
    \end{itemize}
\end{prob}
\begin{sol}
    \begin{itemize}
        \item[a)] The energy of the signal:
        \begin{align}
            \notag E=&\int_{-\infty}^{+\infty}\abs{x(t)}^2\,dt=\int_{-\infty}^{+\infty}e^{-2\alpha\abs{t}}\cos^2\pi t\,dt\\
            \notag=&2\int_0^{+\infty}e^{-2\alpha t}\cos^2\pi t\,dt\\
            \notag=&\int_0^{+\infty}e^{-2\alpha t}(\cos 2\pi t+1)\,dt\\
            \notag=&\int_0^{+\infty}e^{-2\alpha t}\re[e^{i2\pi t}]\,dt-\left.\frac{1}{2\alpha}e^{-2\alpha t}\right\lvert_{0}^{+\infty}\\
            \notag=&\re\left[\int_0^{+\infty}e^{(-2\alpha+i2\pi)t}\,dt\right]+\frac{1}{2\alpha}\\
            \notag=&\re\left[\left.\frac{1}{-2\alpha+i2\pi}e^{(-2\alpha+i2\pi)t}\right\rvert_0^{+\infty}\right]+\frac{1}{2\alpha}\\
            \notag=&\re\left[\frac{1}{2\alpha-i2\pi}\right]+\frac{1}{2\alpha}\\
            =&\frac{\alpha}{2(\alpha^2+\pi^2)}+\frac{1}{2\alpha}<+\infty.
        \end{align}
        % The average power of the signal:
        % \begin{align}
        %     \notag P=&\lim_{T\rightarrow+\infty}\frac{1}{T}\int_{-T/2}^{T/2}\abs{x(t)}^2\,dt=\lim_{T\rightarrow+\infty}\frac{1}{T}\int_{-T/2}^{T/2}e^{-2\alpha\abs{t}}\cos^2\pi t\,dt\\
        %     \notag=&2\lim_{T\rightarrow+\infty}\frac{1}{T}\int_0^{T/2}e^{-2\alpha t}\cos^2\pi t\,dt\\
        %     \notag=&\lim_{T\rightarrow+\infty}\frac{1}{T}\int_0^{T/2}e^{-2\alpha t}(\cos 2\pi t+1)\,dt\\
        %     \notag=&\lim_{T\rightarrow+\infty}\frac{1}{T}\left\{\int_0^{T/2}e^{-2\alpha t}\re[e^{i2\pi t}]\,dt-\left.\frac{1}{2\alpha}e^{-2\alpha t}\right\rvert_0^{T/2}\right\}\\
        %     \notag=&\lim_{T\rightarrow+\infty}\frac{1}{T}\left\{\re\left[\int_0^{T/2}e^{(-2\alpha+i2\pi)t}\,dt\right]+\frac{1-e^{-\alpha T}}{2\alpha}\right\}\\
        %     \notag=&\lim_{T\rightarrow+\infty}\frac{1}{T}\left\{\re\left[\left.\frac{1}{-2\alpha+i2\pi}e^{(-2\alpha+i2\pi)t}\right\rvert_0^{T/2}\right]+\frac{1}{2\alpha}\left(1-e^{-\alpha T}\right)\right\}\\
        %     \notag=&\lim_{T\rightarrow+\infty}\frac{1}{T}\left\{\re\left[\frac{1-e^{(-\alpha+i\pi)T}}{2\alpha-i2\pi}\right]+\frac{1}{2\alpha}\left(1-e^{-\alpha T}\right)\right\}\\
        %     \notag=&\lim_{T\rightarrow+\infty}\frac{1}{T}\left\{\re\left[\frac{\left[1-e^{-\alpha T}(\cos\pi T+i\sin\pi T)\right](2\alpha+i2\pi)}{4(\alpha^2+\pi^2)}\right]+\frac{1}{2\alpha}\left(1-e^{-\alpha T}\right)\right\}\\
        %     =&\lim_{T\rightarrow+\infty}\frac{1}{T}\left\{\frac{\alpha\left[1-e^{-\alpha T\cos\pi T}\cos\pi T\right]+\pi e^{-\alpha T\sin\pi T}\sin\pi T}{2(\alpha^2+\pi^2)}+\frac{1}{2\alpha}\left(1-e^{-\alpha T}\right)\right\}\\
        %     =&0.
        % \end{align}
        Therefore, the signal is \textbf{an energy signal}.
        \item[b)] The energy of the signal:
        \begin{align}
            \notag E=&\int_{-\infty}^{+\infty}\abs{x(t)}^2\,dt=\int_{-\infty}^{+\infty}\abs{\Pi(t-3)\cos 3\pi t}^2\,dt\\
            \notag=&\int_{5/2}^{7/2}\cos^23\pi t\,dt\\
            \notag=&\frac{1}{2}\int_{5/2}^{7/2}(\cos 6\pi t+1)\,dt\\
            \notag=&\frac{1}{2}\left.\left(\frac{1}{6\pi}\sin 6\pi t+t\right)\right\rvert_{5/2}^{7/2}\\
            \notag=&\frac{1}{2}\left[\frac{\sin(7/2)-\sin(5/2)}{6\pi}+1\right]<+\infty.
        \end{align}
        Therefore, the signal is \textbf{an energy signal}.
        \item[c)] The average power of the signal is
        \begin{align}
            \notag P=&\lim_{T\rightarrow+\infty}\frac{1}{T}\int_{-T/2}^{T/2}\abs{x(t)}^2\,dt=\lim_{T\rightarrow+\infty}\frac{1}{T}\int_{-T/2}^{T/2}\abs{t}^2\,dt\\
            \notag=&2\lim_{T\rightarrow+\infty}\frac{1}{T}\int_0^{T/2}t^2\,dt\\
            =&2\lim_{T\rightarrow+\infty}\frac{T^2}{24}=+\infty.
        \end{align}
        Therefore, the signal is \textbf{neither an energy signal nor a power signal}.
        \item[d)] The signal function can be reexpressed as
        \begin{align}
            x(t)=1.
        \end{align}
        The average power of the signal:
        \begin{gather}
            P=\lim_{T\rightarrow+\infty}\frac{1}{T}\int_{-T/2}^{T/2}\abs{x(t)}^2\,dt=\lim_{T\rightarrow+\infty}\frac{1}{T}\int_{-T/2}^{T/2}\,dt=1,\\
            \Longrightarrow 0<P<+\infty.
        \end{gather}
        Therefore, the signal is \textbf{a power signal}.
        \item[e)] The average power of the signal:
        \begin{align}
            \notag P=&\lim_{T\rightarrow+\infty}\frac{1}{T}\int_{-T/2}^{T/2}\abs{x(t)}^2\,dt=\lim_{T\rightarrow+\infty}\frac{1}{T}\int_{-T/2}^{T/2}\abs{e^{j6\pi t}u(t)}^2\,dt\\
            \notag=&\lim_{T\rightarrow+\infty}\frac{1}{T}\int_0^{T/2}\abs{e^{j6\pi t}}^2\,dt\\
            \notag=&\lim_{T\rightarrow+\infty}\frac{1}{T}\int_0^{T/2}\,dt\\
            =&\frac{1}{2}.
        \end{align}
        \begin{align}
            \Longrightarrow 0<P<+\infty.
        \end{align}
        Therefore, the signal is \textbf{a power signal}.
    \end{itemize}
\end{sol}

\begin{prob}[20 pts]
    Calculate \textbf{the Fourier transform and the energy} of the following signals.
    \begin{itemize}
        \item[a)] $x_1(t)=5\sinc(2t)e^{j2\pi 3t}$
        \item[b)] $x_2(t)=\sinc^2(t-1)$
        \item[c)] $x_a(t)=x_1(t)+x_2(-t)$
        \item[d)] $x_b(t)=x_1(-t)+x_2(t)$
        \item[e)] $x_c(t)=2x_1(t)\cos 6\pi t+x_2(t)e^{j6\pi t}$
    \end{itemize}
\end{prob}
\begin{sol}

\end{sol}

\begin{prob}[10 pts]
    Calculate the convolution of the following signal.
    \item[a)] $y(t)=e^{-\abs{t}}*\Pi(t-2)$
    \item[b)] $y(t)=\sgn(t)*\Lambda(t-2)$
\end{prob}
\begin{sol}

\end{sol}

\begin{prob}[20 pts]
    Calculate the Fourier transform of the following periodic signal.
    \begin{itemize}
        \item[a)] $\sum_{n=-\infty}^{\infty}\Lambda(\frac{t}{2}-2n)$
        \item[b)] $\left[\sum_{n=-\infty}^{\infty}\delta(t-2n)\right]*\left[\Pi(\frac{t}{2})\cos(2\pi t)\right]$
    \end{itemize}
\end{prob}
\begin{sol}

\end{sol}

\begin{prob}[15 pts]
    \textbf{Determine} the range of permissible cutoff frequencies for the ideal lowpass filter used to reconstruct the following signal
    \[
        x(t)=4\cos^2(200\pi t)\cos(1000\pi t)
    \]
    Which is sampled at $2000$ samples per second. Sketch $X(f)$ and $X_{\delta}(f)$ (spectrum after the sampling). Find the minimum allowable sampling frequency.
\end{prob}
\begin{sol}

\end{sol}

\begin{prob}
    \begin{itemize}
        \item[1)] Express the spectrum $Y(f)$ of
        \[
            y(t)=x(t)\cos(400\pi t)+\hat{x}(t)\sin(400\pi t)
        \]
        using the spectrum $X(f)$ of $x(t)$, where $\hat{x}(t)$ is the Hilbert transform of $x(t)$. (5 pts)
        \item[2)] if $x(t)=2\cos(60\pi t)$, sketch $Y(f).$
    \end{itemize}
\end{prob}
\begin{sol}

\end{sol}

\begin{prob}
    Consider $x(t)=2\cos(60\pi t)$, the reference frequency $f_0=40$ Hz. Calculate the following signals.
    \begin{itemize}
        \item[a)] The Hilbert transform of $x(t)$, i.e., $\hat{x}(t)$.
        \item[b)] The analytic signal $x_p(t)$.
        \item[c)] The convex envelope of $x(t)$, i.e., $\tilde{x}(t)$.
        \item[d)] The inphase and quadrature component of $x(t)$, i.e., $x_R(t)$ and $x_I(t)$.\\
        (Please refer to Lecture 2, Slide 36 or Page 88 of reference textbook, we will learn this in the next class. I am sorry for the lagging.)
        \item[e)] Determine and plot the spectrum of the following signals:
        \begin{itemize}
            \item[i.] $x_1(t)=\frac{2}{3}x(t)+\frac{1}{3}j\hat{x}(t)$
            \item[ii.] $x_2(t)=\left[\frac{1}{5}x(t)+\frac{4}{5}j\hat{x}(t)\right]e^{j2\pi f_0t}$
        \end{itemize}
    \end{itemize}
\end{prob}
\begin{sol}

\end{sol}
\end{document}