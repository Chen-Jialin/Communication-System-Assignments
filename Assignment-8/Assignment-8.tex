\documentclass{assignment}
\UsingEnglish
\ProjectInfos*{Intro to Communication System}{EE140}{Fall, 2020}{Assignment 8}{Due time : 10:15, Nov 27, 2020 (Friday)}{陈稼霖}{45875852}
\begin{document}
\begin{prob}[2.11]
    Proof of the Kraft inequality for uniquely decodable codes.
    \begin{itemize}
        \item[(a)] Assume a uniquely decodable code has lengths $l_1,\cdots l_M$. In order to show that $\sum_j2^{-l_j}\leq 1$, demonstrate the following identity for each integer $n\geq 1$:
        \[
            \left[\sum_{j=1}^M2^{-l_j}\right]^n=\sum_{j_1=1}^M\sum_{j_2=1}^M\cdots\sum_{j_n=1}^M2^{-(l_{j_1}+l_{j_2}+\cdots l_{j_n})}.
        \]
        \item[(b)] Show that there is one term on the right for each concatenation of $n$ codewords (i.e. for the encoding of one $n$-tuple $\bm{x}^n$) where $l_{j_1}+l_{j_2}+\cdots+l_{j_n}$ is the aggregate length of that concatenation.
        \item[(c)] Let $A_i$ be the number of concatenations which have overall length $i$ and show that
        \[
            \left[\sum_{j=1}^M2^{-l_j}\right]^n=\sum_{i=1}^{nl_{max}}A_i2^{-i}.
        \]
        \item[(d)] Using the unique decodablility, upperbound each $A_i$ and show that
        \[
            \left[\sum_{j=1}^M2^{-l_j}\right]^n\leq nl_{\max}.
        \]
        \item[(e)] By taking the $n$th root and letting $n\rightarrow\infty$, demonstrate the Kraft inequality.
    \end{itemize}
\end{prob}
\begin{pf}
    \begin{itemize}
        \item[(a)] 
        \begin{align}
            \left[\sum_{j=1}M2^{-l_j}\right]^n=\left[\sum_{j_1=1}^M2^{-l_{j_1}}\right]\left[\sum_{j_2=1}^M2^{-l_{j_2}}\right]\cdots\left[\sum_{j_n=1}^M2^{-l_{j_n}}\right]=\sum_{j_1=1}^M\sum_{j_2=1}^M\cdots\sum_{j_n=1}^M2^{-(l_{j_1}+l_{j_2}+\cdots+l_{j_n})},\quad\forall n\geq 1.
        \end{align}
        \item[(b)] For each concatenation of $n$ codewords $\bm{x}^n$, the length of the $k$th codeword is $l_{j_k}$ and there is $n$ codewords in total ($1\leq k\leq n$), so the aggregate length of the concatenation is $l_{j_1}+l_{j_2}+\cdots+l_{j_n}$.
        \item[(c)] Using the conclusion we obtained in (b), we have
        \begin{align}
            \left[\sum_{j=1}^M2^{-l_j}\right]^n=\sum_{\bm{x}^n}e^{-(\text{length of }\bm{x}^n)}=\sum_{i=1}^{nl_{\max}}A_i2^{-i}.
        \end{align}
        \item[(d)] Because of unique decodablility (i.e. each concatenation should be different),
        \begin{align}
            A_i\leq 2^i,\quad\forall i.
        \end{align}
        Thus,
        \begin{align}
            \label{P-1-d}
            \left[\sum_{j=1}^M2^{-l_j}\right]^n=\sum_{i=1}^{nl_{\max}}A_i2^{-i}\leq\sum_{i=1}^{nl_{\max}}2^i2^{-i}=\sum_{i=1}^{nl_{\max}}1=nl_{\max}.
        \end{align}
        \item[(e)] Taking the $n$th root of equation \eqref{P-1-d}, we have
        \begin{align}
            \sum_{j=1}^M2^{-l_j}\leq[nl_{\max}]^{1/n}.
        \end{align}
        Letting $n\rightarrow\infty$, we have
        \begin{align}
            \sum_{j=1}^M2^{-l_j}\leq\lim_{n\rightarrow\infty}[nl_{\max}]^{1/n}=1,
        \end{align}
        which is the Kraft inequality.
    \end{itemize}
\end{pf}

\begin{prob}[2.12]
    A source with an alphabet size of $M=\abs{\mathcal{X}}=4$ has a symbol probabilities $\{1/3,1/3,2/9,1/9\}$.
    \begin{itemize}
        \item[(a)] Use the Huffman algorithm to find an optimal prefix free code for this source.
        \item[(b)] Use the Huffman algorithm to find another optimal prefix-free code with a different set of lengths.
        \item[(c)] Find another prefix-free code that is optimal but cannot result from using the Huffman algorithm.
    \end{itemize}
\end{prob}
\begin{sol}
\end{sol}

\begin{prob}[2.14]
    Consider a source with $M$ equiprobable symbols.
    \begin{itemize}
        \item[(a)] Let $k=\lceil\log M\rceil$. Show that, for a Huffman code, the only possible codeword lengths are $k$ and $k-1$.
        \item[(b)] As a function of $M$, find how many codewords have length $k=\lceil\log M\rceil$. What is the expected codeword length $\bar{L}$ in bits per source code?
        \item[(c)] Define $y=M/2^k$. Express $\bar{L}-\log M$ as a function of $y$. Find the maximum value of this function over $1/2<y\leq 1$. This illustrates that the entropy bound, $\bar{L}=H[X]+1$, is rather loose in this equiprobable case.
    \end{itemize}
\end{prob}
\begin{sol}
\end{sol}

\begin{prob}[2.21]
    A discrete memoryless source emits iid random symbols $X_1,X_2,\cdots$ Each random symbol $X$ has the symbols $\{a,b,c\}$ with probabilities $\{0.5,0.4,0.1\}$, respectively.
    \begin{itemize}
        \item[(a)] Find the expected length $\bar{L}_{\min}$ of the best variable-length prefix-free code for $X$.
        \item[(b)] Find the expected length $\bar{L}_{\min,2}$, normalized to bits per symbol, of the best variable-length prefix-free code for $X^2$.
        \item[(c)] Is it true that for any DMS, $\bar{L}_{\min}\geq\bar{L}_{\min,2}$? Explain your answer.
    \end{itemize}
\end{prob}
\begin{sol}
\end{sol}

\begin{prob}[2.33]
    Perform an LZ77 parsing of the string \uline{00011101}0010101100. Assume a window of length $W=8$; the initial window is underlined above. You should parse the reset of the string using the Lempel-Ziv algorithm.
\end{prob}
\begin{sol}
\end{sol}

\begin{prob}[4.35 Aliasing]
    The following exercise is designed to illustrate the sampling of an approximately baseband waveform. To avoid messy computation, we look at a waveform baseband-limited to $3/2$ which is sampled at rate $1$ (i.e. sampled at only $1/3$ the rate that it should be sampled at). In particular, let $u(t)=\sinc(3t)$.
    \begin{itemize}
        \item[(a)] Sketch $\hat{u}(f)$. Sketch the function $\hat{v}_m(t)=\rect(f-m)$ for each integer $m$ such that $v_m(f)\neq 0$. Note that $\hat{u}(f)=\sum_m\hat{v}_m(f)$.
        \item[(b)] Sketch the inverse transforms $v_m(t)$ (real and imaginary parts if complex).
        \item[(c)] Verify directly from the equations that $u(t)=\sum v_m(t)$. [Hint. This is easier if you express the sine part of the sinc function as a sum of complex exponentials.]
        \item[(d)] Verify the sinc-weighted sinusoid expansion, (4.73). (There are only three nonzero terms in the expansion.)
        \item[(e)] For the approximation $s(t)=u(0)\sinc(t)$, find the energy in the difference between $u(t)$ and $s(t)$ and interpret the terms.
    \end{itemize}
\end{prob}
\begin{sol}
\end{sol}
\end{document}